% Include settings and imports
%---------------------------------------------------------------------
% Imports and settings
%---------------------------------------------------------------------

\documentclass[12pt, a4paper]{scrartcl}

% Language and encoding
\usepackage[utf8]{inputenc}
\usepackage[USenglish]{babel}
\usepackage[T1]{fontenc}

% Hyperrefs / links
\usepackage{hyperref}

% Appendix
\usepackage[title]{appendix}

% Used to restrict floating of tables/illustrations
\usepackage{float}
\usepackage{placeins}

% Footnotes + captions
\usepackage{caption} 
\usepackage{footnote}
%\usepackage{ftnxtra}

% Tables
\usepackage{tabulary}
\usepackage{tabularx}
\usepackage{multirow}
\renewcommand{\arraystretch}{1.4} % increase spacing for table rows
\usepackage{longtable}

% Biblatex: use stylefiles *.cbx, *.bbx
\usepackage[citestyle=dhbw_ibim,bibstyle=dhbw_ibim,sorting=nty,block=space,firstinits=true]{biblatex}
\providecommand{\BIBand}{and}

% Bibliography file
\bibliography{software_documentation} 

% Abbreviations
\usepackage{nomencl}
\renewcommand{\nomname}{List of Abbreviations}
\setlength{\nomlabelwidth}{.25\hsize}
\renewcommand{\nomlabel}[1]{#1 \dotfill}
\makenomenclature

% Listings
% see http://en.wikibooks.org/wiki/LaTeX/Packages/Listings
\usepackage{listings}
\lstset{numbers=left,captionpos=b,basicstyle=\footnotesize}

% The Euro Symbol Package for LaTeX
\usepackage[right]{eurosym}

%--------------------------------------------------------------------
% LAYOUT
%--------------------------------------------------------------------

% Abstand der Absätze 1.5ex:
\usepackage{setspace}

% Kopfzeilenpaket:
\usepackage{scrpage2}

% für das Einbinden von Grafiken
\usepackage{graphicx}
\usepackage{wrapfig}

%Definition der Ränder
\usepackage[paper=a4paper,left=35mm,right=15mm,top=25mm,bottom=20mm]{geometry}

%Abstand der Fußnoten
\deffootnote{10pt}{10pt}{\textsuperscript{\thefootnotemark\ }}

% Regeln, bis zu welcher Tiefe (section,subsection,subsubsection) 
% Überschriften angezeigt werden sollen (Anzeige der Überschriften im
% Verzeichnis / Anzeige der Nummerierung)
\setcounter{tocdepth}{3}
\setcounter{secnumdepth}{3}

% Keine "Schusterjungen"
\clubpenalty = 10000
% Keine "Hurenkinder"
\widowpenalty = 10000 
\displaywidowpenalty = 10000

\renewcommand{\rmdefault}{phv} % Arial
\renewcommand{\sfdefault}{phv} % Arial




%--------------------------------------------------------------------
% Document information
%--------------------------------------------------------------------


% Substitutions, only edit the contents here. These commands are used
% elsewhere in the paper so that typing is reduced to a minimum
\newcommand{\papertitle}{PAPERS TITLE}
\newcommand{\paperauthor}{YOUR NAME}

\newcommand{\dhbwprogram}{STUDY PROGRAM}
\newcommand{\dhbwcourse}{COURSE NAME}

% Document itself
\title{\papertitle}
\author{\paperauthor}
\date{\today}

% for Distiller pdftex:
\hypersetup{ 
	pdfauthor={\paperauthor},
	pdftitle={\papertitle},
	pdfsubject={1st Project Paper - Program:  \dhbwprogram| Course: \dhbwcourse},
	pdfkeywords = some keywords separated by whitespace,
	linkcolor={black},
	pdfview=FitV,     % FitH
	pdfstartview=FitV,
	urlcolor=black,   
	breaklinks=true,   
	citebordercolor=0 0 0, 
	filebordercolor=0 0 0,
	linkbordercolor=0 0 0,
	menubordercolor=0 0 0,
	urlbordercolor=0 0 0,
	pdfhighlight=/I,
	pdfborder=0 0 0,   % no boxes for hyperrefs
	  } 



%-------------------
% Begin document
%-------------------
\begin{document}

% roman numerals
\renewcommand{\thepage}{\Roman{page}}

\pagestyle{scrheadings}

% page numbers centered on top:
\chead{\pagemark}
\cfoot{}



%----------------------------------------------------------------------------
% Title Page
%----------------------------------------------------------------------------

% no page numbering
\thispagestyle{empty}
% include title page
\begin{titlepage}
\vspace*{\fill}
\begin{center}
\vspace{3mm}


\textbf{ % Title of paper (defined in header.tex)
	\papertitle
	} \\
\vspace{1.5cm}


\papertype \\ % Type of paper (defined in header.tex)
\vspace{1cm}


provided on DD/MM/YYYY \\ % Date of provision
\vspace{1cm}
School of: \schoolof  \\
\vspace{0.6cm}

Program: \studyprogram \\ % Program (defined in header.tex)
\vspace{0.6cm}
Course: \studycourse \\ % Course (defined in header.tex)
		
\vspace{2.5cm}
by \\
\paperauthor % Author (defined in header.tex)
\vspace{3cm}

\begin{tabular}[h]{ll}

Training Company:				&	BW Cooperative State University\\
								&	Stuttgart:\\ 
 								&	\\
Company							&	Prof. Dr. X\\
Max Mustermann					&	\\
Job Role				  		&	\\
Department						&	\\
								&	\\
\rule{0.5\textwidth}{1pt}		&	\\				
Signature						&	\\		
								&	\\
%uncomment lines below if your paper contains confidential information

%Confidential			&	\\
%\multicolumn{2}{p{15cm}}{The content of the paper must not be made available to third
%parties without approval of the training company.} \\
\end{tabular}
\end{center}
\vspace*{\fill}
\end{titlepage}

\renewcommand{\baselinestretch}{1.5}\normalsize


%----------------------------------------------------------------------------
% Table of Contents
%----------------------------------------------------------------------------
\tableofcontents
\newpage


%----------------------------------------------------------------------------
% Abbreviations
%----------------------------------------------------------------------------
% List needs to be indexed after each change.
% This is done by executing the following command:
% ~$ makeindex [filename].nlo -s nomencl.ist -o [filename].nls
\nomenclature{API}{Applicable Programming Interface}%
\nomenclature{DVCS}{Distributed Version Control System}%
\nomenclature{LDAP}{Light Directory Access Protocol}%
\nomenclature{PEP}{Python Enhancement Proposal}%
\nomenclature{SSH}{Secure Shell}%
\nomenclature{SVN}{Subversion}%
\printnomenclature
\addcontentsline{toc}{section}{List of Abbreviations}
\newpage


%----------------------------------------------------------------------------
% List Of Tables
%----------------------------------------------------------------------------
\listoftables
\addcontentsline{toc}{section}{List of Tables}
\newpage


%----------------------------------------------------------------------------
% List of Figures
%----------------------------------------------------------------------------
\listoffigures
\addcontentsline{toc}{section}{List of Figures}
\newpage

% Arabic numerals for page numbering
\renewcommand{\thepage}{\arabic{page}}

% Set page number to 1: 
\setcounter{page}{1} 


%----------------------------------------------------------------------------
% Text of Paper
%----------------------------------------------------------------------------
% usage of cite commands as follows:
% \footcite[ prenote ][ postnote/page ]{ key }



\section{Introduction}
%=========================================================================
% "In the introduction, the problem and the thus resulting objective of the 
% paper must be precisely formulated and the procedure explained. Any 
% necessary differentiations must be made here. Instead of the headline
% 'Introduction' it is also possible to use a different, more expressive
% headline." (taken from the citation regulations of DHBW)

Lorem ipsum dolor sit amet, consectetur adipiscing elit. Phasellus placerat
justo et velit porta tempus. Vestibulum quis elit erat. Fusce in dictum neque.
Nulla turpis sapien, iaculis ut placerat vel, mollis non lectus. Nam eu tortor
id mi vestibulum sodales. Fusce vel felis eget felis ornare sollicitudin. Ut
sit amet magna vitae nibh tristique blandit. Fusce congue urna quis urna
sollicitudin vitae gravida ligula ultrices. Lorem ipsum dolor sit amet,
consectetur adipiscing elit. Morbi urna ligula, mollis eget adipiscing id,
condimentum ac purus. Morbi nec nulla non eros molestie fermentum. Proin
euismod rutrum magna et venenatis. 


\newpage

\section{Exemplary content}
%=========================================================================


\subsection{Listings}
%--------------------------------------------------------------------------

Read \url{https://en.wikibooks.org/wiki/LaTeX/Packages/Listings} for more
information on the listings package.\\

Source code can be included from external files:\\

\lstinputlisting[language=Python, caption={Python function},
				 label=lst:pyfunction]{lst/example-listing.py}

Or as raw input in the paper:\\

\begin{lstlisting}[breaklines=true,
				   frame=single,
				   caption={WebDAV configuration in
					/etc/apache2/mods-available/dav\_svn.conf},
				   label=lst:davsvn]
<Location /svn/umbrella>
  DAV svn
  SVNPath /var/svn/umbrella
  AuthType Basic
  AuthName "Subversion Repository"
  AuthUserFile /etc/apache2/dav_svn.passwd
  Require valid-user
</Location>
\end{lstlisting}


\lstset{tabsize=2,language=bash}
\begin{lstlisting}[breaklines=true,frame=single,caption={Shellscript for building the software documentation in HTML and PDF},label=lst:builddocs]
#!/bin/bash
sphinx-apidoc -f -o source/api/apps ../umbrella/apps
echo "building html docs..." > build.log
annotate-output make html >> build.log
echo "building pdf doc with latex..." >> build.log
annotate-output make latexpdf >> build.log
\end{lstlisting}


\subsection{Tables}
%--------------------------------------------------------------------------
\label{sec:tables}

There are several packages / environments available to style tables.
An overview is available on \url{https://en.wikibooks.org/wiki/LaTeX/Tables}.



\begin{table}[H]
\begin{center}
  \begin{tabular}{p{2cm}|p{8cm}}
	Priority &	Documentation artifact\\\hline
	1. &	General project information \\
	2. &	Requirements \\
	3. &	Low level documentation\\
	4. &	System architecture\\
  \end{tabular}
  \captionof{table}{Priority of artifacts (descending)}
  \label{tab:importance_artifacts}
\end{center}
\end{table}


\begin{table}[H]
\begin{center}
\rowcolors{1}{gray}{white}

  \begin{tabular}{lll}
    odd     & odd   & odd \\
    even    & even  & even\\
    odd     & odd   & odd \\
    even    & even  & even\\
  \end{tabular}
  \captionof{table}{Table with colored rows}
  \label{tab:colored}
\end{center}
\end{table}


\subsection{Graphics}
%--------------------------------------------------------------------------
\label{sec:graphics}

\begin{figure}[H]
\begin{center}
	\fbox{
	\includegraphics[width=\textwidth]{img/placeholder1.jpg}
	}
	\captionof{figure}{Caption text}
	\label{fig:placeholder}
\end{center}
\end{figure}


\begin{figure}[h]
\setlength{\unitlength}{0.14in}% selecting unit length
\centering
% 32 length units wide, and 15 units high.
\begin{picture}(32,15)

\put(3,4){\framebox(6,3){$H_{B}(q)$}}
\put(13,4){\framebox(6,3){$N[\cdot]$}}
\put(23,4){\framebox(6,3){$H_{C}(q)$}}
\put(0,5.5){\vector(1,0){3}}\put(9,5.5){\vector(1,0){4}}
\put(19,5.5){\vector(1,0){4}}\put(29,5.5){\vector(1,0){3}}
\put(-1,6.5) {$u(k)$}\put(30,6.5) {$y(k)$} \put(9.5,6.5)
{$x_{B}(k)$}\put(19.5,6.5) {$x_{C}(k)$}
\end{picture}
	\caption[LNL Block Oriented Model Structure]{An LNL Block Oriented Model
	Structure (taken from
		\url{http://www.iit.edu/graduate_college/academic_affairs/pdfs/FigureHelp1.pdf})}
\label{fig:lnlbl}
\end{figure}

Other resources for graphics:
\begin{description}
  \item[graphviz / dot] \url{http://www.graphviz.org/Resources.php}
  \item[PyX output via Python] \url{http://www.texample.net/weblog/2008/oct/24/embedding-python-latex/}
\end{description}

You can set FloatBarriers to make sure that graphics are placed inside one
chapter.

\FloatBarrier %nothing should float out of this subsection
\newpage





\subsection{Citation examples and references}
%--------------------------------------------------------------------------
\label{sec:citation}

\subsubsection{Citations}
%------------------------

%citing examples:
While \citeauthor{tilley_1992} stresses out the general importance of documented
system architecture\footcite[Cf.][]{tilley_1992}, 
\citeauthor{cioch_96} mentions that relevance depends on the
maintainers stage of experience (from new comer to expert; after some years of
working on the system). According to their study new comers need a general
overview of the system and experts need low-level documentation such as API
descriptions.\footcite[Cf.][]{cioch_96}

According to a survey by \citeauthor{forward_relevance_2002} among managers and
maintainers, specification documents are most consulted whereas low level
documents are least consulted\footcite[Cf.][28-30]{forward_relevance_2002}.

\citeauthor{grubb2003software} map information needs to maintainers roles
in the software project. Analysts need to have a global view of the system and
know the requirements. Designers need architectural understanding and
information about low-level components. Finally programmers need a deep
understanding of the source code and a higher level view - comparable to level
two and three above.\footcite[Cf.][103-106]{grubb2003software}

A survey by \citeauthor{de_souza_study_2005} offers a more differentiated view.
Software maintainers were asked to answer how often they use different documents
and how important they think these documents are. Answers of maintainers who
said they work in projects with structured analysis paradigm (70 maintainers)
were separated of those who said they work in projects with an object-oriented
approach (54 maintainers).\footcite[Cf.][69-74]{de_souza_study_2005}
According to the results of the survey, the source code itself is most important
and comments get ranked second. Contrary to the results of \citeauthor{forward_relevance_2002}
specification documents are only ranked on places seven to
ten\footcite[Cf.][98]{huang_towards_2003}. One reason for these contrary results
might be different project sizes, stages and environments.

\subsubsection{References}
%--------------------------

Reference to table \ref{tab:importance_artifacts}.\\
Reference to figure \ref{fig:placeholder}.
Reference to section \ref{sec:conclusion}.


\newpage


\section{Conclusion}
%=========================================================================
\label{sec:conclusion}

Fusce id magna leo, in porta lectus. Aenean convallis magna sit amet ligula
porttitor vehicula. Morbi pulvinar justo sed sapien pellentesque vitae viverra
dui hendrerit. Vivamus nulla nunc, imperdiet congue euismod ultricies,
sollicitudin eu felis. Pellentesque commodo risus eu quam facilisis et semper
dolor aliquam. Vivamus neque magna, mattis vel aliquet eu, vulputate et leo.
Proin congue dui eget dolor tempus vel congue magna ultrices. Aliquam ut elit
ac lorem molestie adipiscing. Duis at nulla et lorem condimentum faucibus.
Integer commodo turpis id erat volutpat consectetur. Mauris tempor gravida
dui, vitae hendrerit quam laoreet sed. Vivamus eu lorem vel massa pulvinar
pretium. Aenean dignissim porta arcu, ac pretium tortor placerat porttitor. 




%-----------------------------------
%END CONTENT
%-----------------------------------


% include footer with appendix, affidavit,...
%----------------------------------------------------------------------------
% APPENDIX
%----------------------------------------------------------------------------
\newpage \begin{appendices} %sets the appendix environment and resets the
							% section counters
				% \renewcommand{\appendixtocname}{List of appendices}
				% \renewcommand{\appendixpagename}{List of appendices}
\appendixtocon %adds an 'Appendices' entry to the toc
\renewcommand*\appendixpagename{\Large Appendices}  %renewal needed to set
													% proper size of title
\appendixpage %prints the title on the page

\subsection*{List of Appendices}
% TODO: create list of appendices with same style as lot or lof


\begin{subappendices}
% \appendixtocoff cant be used here.
% Need to turn toc off for whole environment and manually set the toc entry
% with \addcontentsline{toc}{subsection}{Preface}

\renewcommand{\setthesubsection}{\arabic{subsection}:}%


\subsection{something}
kskfslfrlerg wre gergh  ht rh r h rh   hrthrhrtjukseweffre htzkjtsht erthzrj
kskfslfrlerg wre gergh  ht rh r h rh   hrthrhrtjukseweffre htzkjtsht erthzrj
kskfslfrlerg wre gergh  ht rh r h rh   hrthrhrtjukseweffre htzkjtsht erthzrj
kskfslfrlerg wre gergh  ht rh r h rh   hrthrhrtjukseweffre htzkjtsht erthzrj
kskfslfrlerg wre gergh  ht rh r h rh   hrthrhrtjukseweffre htzkjtshtks
kfslfrlerg\\
wre gergh  ht rh r h rh   hrthrhrtjukseweffre htzkjtsht erthzrj erthzrj

\subsection{something else}
najsjdn fwfwegt eg r th er5h z5er6h  rhg 

\end{subappendices}
\end{appendices}
%----------------------------------------------------------------------------
% BIBLIOGRAPHY
%----------------------------------------------------------------------------
% bibliography file is set in header.tex

\newpage
\addsec{Lists of references} %addsec: komaskript befehl - keine Nummerierung

% Set headings for bibliographies
\defbibheading{literatur}{\subsection*{List of Literature}}
\defbibheading{www}{\subsection*{List of Internet and Intranet Resources}}

% Show bibliographies
\printbibliography[nottype=online,heading=literatur]
% warning: includes ALL types but online.
% Check your bibfile to avoid wrong entries here
\printbibliography[type=online,heading=www]


%----------------------------------------------------------------------------
% Affidavit
%----------------------------------------------------------------------------
\newpage
\pagestyle{empty}

\section*{Affidavit}

\vspace{0.5cm}

"I hereby declare\\

\begin{enumerate}
	\item that I completed my project paper without external help;
	\item that I indicated the use of literal citations from literature as
	      well as the use of ideas of other authors at the respective
	      places in the paper.
	\item that I have not presented my project paper for any other
	      examination.
\end{enumerate}

I am aware of the fact that an untrue declaration will have legal
consequences."
\\ \\ \\ \\
\_\_\_\_\_\_\_\_\_\_\_\_\_\_\_\_\_\_\_\_\_\_\_\_\\
PLACE, DD/MM/YYYY


\end{document}

