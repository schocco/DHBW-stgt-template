% Include settings and imports
%---------------------------------------------------------------------
% Imports and settings
%---------------------------------------------------------------------

\documentclass[12pt, a4paper]{scrartcl}

% Language and encoding
\usepackage[utf8]{inputenc}
\usepackage[USenglish]{babel}
\usepackage[T1]{fontenc}

% Hyperrefs / links
\usepackage{hyperref}

% Appendix
\usepackage[title]{appendix}

% Used to restrict floating of tables/illustrations
\usepackage{float}
\usepackage{placeins}

% Footnotes + captions
\usepackage{caption} 
\usepackage{footnote}
%\usepackage{ftnxtra}

% Tables
\usepackage{tabulary}
\usepackage{tabularx}
\usepackage{multirow}
\renewcommand{\arraystretch}{1.4} % increase spacing for table rows
\usepackage{longtable}

% Biblatex: use stylefiles *.cbx, *.bbx
\usepackage[citestyle=dhbw_ibim,bibstyle=dhbw_ibim,sorting=nty,block=space,firstinits=true]{biblatex}
\providecommand{\BIBand}{and}

% Bibliography file
\bibliography{software_documentation} 

% Abbreviations
\usepackage{nomencl}
\renewcommand{\nomname}{List of Abbreviations}
\setlength{\nomlabelwidth}{.25\hsize}
\renewcommand{\nomlabel}[1]{#1 \dotfill}
\makenomenclature

% Listings
% see http://en.wikibooks.org/wiki/LaTeX/Packages/Listings
\usepackage{listings}
\lstset{numbers=left,captionpos=b,basicstyle=\footnotesize}

% The Euro Symbol Package for LaTeX
\usepackage[right]{eurosym}

%--------------------------------------------------------------------
% LAYOUT
%--------------------------------------------------------------------

% Abstand der Absätze 1.5ex:
\usepackage{setspace}

% Kopfzeilenpaket:
\usepackage{scrpage2}

% für das Einbinden von Grafiken
\usepackage{graphicx}
\usepackage{wrapfig}

%Definition der Ränder
\usepackage[paper=a4paper,left=35mm,right=15mm,top=25mm,bottom=20mm]{geometry}

%Abstand der Fußnoten
\deffootnote{10pt}{10pt}{\textsuperscript{\thefootnotemark\ }}

% Regeln, bis zu welcher Tiefe (section,subsection,subsubsection) 
% Überschriften angezeigt werden sollen (Anzeige der Überschriften im
% Verzeichnis / Anzeige der Nummerierung)
\setcounter{tocdepth}{3}
\setcounter{secnumdepth}{3}

% Keine "Schusterjungen"
\clubpenalty = 10000
% Keine "Hurenkinder"
\widowpenalty = 10000 
\displaywidowpenalty = 10000

\renewcommand{\rmdefault}{phv} % Arial
\renewcommand{\sfdefault}{phv} % Arial




%--------------------------------------------------------------------
% Document information
%--------------------------------------------------------------------


% Substitutions, only edit the contents here. These commands are used
% elsewhere in the paper so that typing is reduced to a minimum
\newcommand{\papertitle}{PAPERS TITLE}
\newcommand{\paperauthor}{YOUR NAME}

\newcommand{\dhbwprogram}{STUDY PROGRAM}
\newcommand{\dhbwcourse}{COURSE NAME}

% Document itself
\title{\papertitle}
\author{\paperauthor}
\date{\today}

% for Distiller pdftex:
\hypersetup{ 
	pdfauthor={\paperauthor},
	pdftitle={\papertitle},
	pdfsubject={1st Project Paper - Program:  \dhbwprogram| Course: \dhbwcourse},
	pdfkeywords = some keywords separated by whitespace,
	linkcolor={black},
	pdfview=FitV,     % FitH
	pdfstartview=FitV,
	urlcolor=black,   
	breaklinks=true,   
	citebordercolor=0 0 0, 
	filebordercolor=0 0 0,
	linkbordercolor=0 0 0,
	menubordercolor=0 0 0,
	urlbordercolor=0 0 0,
	pdfhighlight=/I,
	pdfborder=0 0 0,   % no boxes for hyperrefs
	  } 



%-------------------
% Begin document
%-------------------
\begin{document}

% roman numerals
\renewcommand{\thepage}{\Roman{page}}

\pagestyle{scrheadings}

% page numbers centered on top:
\chead{\pagemark}
\cfoot{}



%----------------------------------------------------------------------------
% Title Page
%----------------------------------------------------------------------------

% no page number
\thispagestyle{empty}

\begin{titlepage}
\vspace*{\fill}
\begin{center}
\vspace{3mm}


\textbf{ % Title of paper (defined in header.tex)
	\papertitle
	} \\
\vspace{1.5cm}


1\textsuperscript{st} Project Paper \\ % Type of paper
\vspace{1cm}


provided on DD/MM/YYYY \\ % Date of provision
\vspace{1cm}
School of: Business  \\
\vspace{0.6cm}

Program: \dhbwprogram \\ % Program (defined in header.tex)
\vspace{0.6cm}
Course: \dhbwcourse \\ % Course (defined in header.tex)
		
\vspace{2.5cm}
by \\
\paperauthor % Author (defined in header.tex)
\vspace{3cm}

\begin{tabular}[h]{ll}

Training Company:				&	BW Cooperative State University\\
						&	Stuttgart:\\ 
 						&	\\
Company						&	Prof. Dr. X\\
Max Mustermann					&	\\
Job Role				  	&	\\
Department					&	\\
						&	\\
\_\_\_\_\_\_\_\_\_\_\_\_\_\_\_\_\_\_\_\_\_\_\_\_&	\\				
Signature					&	\\		
						&	\\
%uncomment lines below if your paper contains confidential information

%Confidential			&	\\
%\multicolumn{2}{p{15cm}}{The content of the paper must not be made available to third
%parties without approval of the training company.} \\
\end{tabular}
\end{center}
\vspace*{\fill}
\end{titlepage}

\renewcommand{\baselinestretch}{1.5}\normalsize


%----------------------------------------------------------------------------
% Table of Contents
%----------------------------------------------------------------------------
\tableofcontents
\newpage


%----------------------------------------------------------------------------
% Abbreviations
%----------------------------------------------------------------------------
% List needs to be indexed after each change.
% This is done by executing the following command:
% ~$ makeindex [filename].nlo -s nomencl.ist -o [filename].nls
\addcontentsline{toc}{section}{List of Abbreviations}

\nomenclature{API}{Applicable Programming Interface}%
\nomenclature{DVCS}{Distributed Version Control System}%
\nomenclature{LDAP}{Light Directory Access Protocol}%
\nomenclature{PEP}{Python Enhancement Proposal}%
\nomenclature{SSH}{Secure Shell}%
\nomenclature{SVN}{Subversion}%
\nomenclature{VCS}{Version Control System}%
\nomenclature{HTML}{Hyper Text Transfer Protocol}%
\nomenclature{PDF}{Portable Document Format}%
\nomenclature{UML}{Unified Modeling Language}%
\nomenclature{PEP}{Python Enhancement Proposal}%
\nomenclature{BSD}{Berkeley Software Distribution}%
\nomenclature{GPL}{GNU General Public License}%
\nomenclature{RTF}{Rich Text Format}%
\printnomenclature
\newpage

%----------------------------------------------------------------------------
% List Of Tables
%----------------------------------------------------------------------------
\addcontentsline{toc}{section}{List of Tables}
\listoftables
\newpage

%----------------------------------------------------------------------------
% List of Figures
%----------------------------------------------------------------------------
\addcontentsline{toc}{section}{List of Figures}
\listoffigures

\newpage

% Arabic numerals for page numbering
\renewcommand{\thepage}{\arabic{page}}

% Set page number to 1: 
\setcounter{page}{1} 


%----------------------------------------------------------------------------
% Text of Paper
%----------------------------------------------------------------------------
% usage of cite commands as follows:
% \footcite[ prenote ][ postnote/page ]{ key }



\section{Introduction}
%======================
% "In the introduction, the problem and the thus resulting objective of the 
% paper must be precisely formulated and the procedure explained. Any 
% necessary differentiations must be made here. Instead of the headline
% 'Introduction' it is also possible to use a different, more expressive
% headline." (taken from the citation regulations of DHBW)

Lorem ipsum dolor sit amet, consectetur adipiscing elit. Phasellus placerat
justo et velit porta tempus. Vestibulum quis elit erat. Fusce in dictum neque.
Nulla turpis sapien, iaculis ut placerat vel, mollis non lectus. Nam eu tortor
id mi vestibulum sodales. Fusce vel felis eget felis ornare sollicitudin. Ut
sit amet magna vitae nibh tristique blandit. Fusce congue urna quis urna
sollicitudin vitae gravida ligula ultrices. Lorem ipsum dolor sit amet,
consectetur adipiscing elit. Morbi urna ligula, mollis eget adipiscing id,
condimentum ac purus. Morbi nec nulla non eros molestie fermentum. Proin
euismod rutrum magna et venenatis. 


\newpage
%--------------------------------------------------------------
\section{Development in dynamic teams with alternating members}
%---------------------------------------------------------------

example listing
\lstinputlisting[language=Python]{lst/example-listing.py}

Morbi tristique nibh ut massa viverra a condimentum elit aliquam. Donec et
mauris ut orci eleifend condimentum vitae a felis. Duis ornare ipsum vel lorem
semper ornare. Mauris sapien diam, interdum non posuere et, mollis vitae
lorem. Quisque feugiat hendrerit mauris, non pretium metus interdum a. In nisi
mi, venenatis vel eleifend vel, gravida non nulla. Ut tincidunt mattis
laoreet. Phasellus vel rhoncus tortor. Praesent sed purus augue, et
ullamcorper erat. Pellentesque ac scelerisque nunc. 


 
\newpage
%-----------------------------------
\section{Software documentation}
%------------------------------------------
\FloatBarrier %nothing should float out of this subsubsection


\subsection{Components of software documentation}
Mauris ut leo ligula. Donec commodo eros a sem viverra a gravida metus
hendrerit. Suspendisse potenti. Suspendisse imperdiet iaculis odio sed
sagittis. Maecenas sem mauris, auctor quis rhoncus quis, auctor nec mauris.
Aenean vestibulum malesuada sapien et posuere. Cras odio erat, gravida ac
venenatis ut, fermentum nec leo.

Donec commodo eros \citetitle{huang_towards_2003}. The
authors identify five levels of granularity which are listed
below\footcite[Cf.][98]{huang_towards_2003}.

\begin{description}
	\item[Level 1 - source code] Documentation on this level contains the
	     source code itself and its associated comments. It helps
	     understanding algorithms and data structures.

	\item[Level 2 - design patterns] These are common programming idioms
	     that are language independent and therefore on a higher abstraction
	     level than source code documentation.

	\item[Level 3 - software architecture] Software architecture, or
	     high-level design, describes the structural characteristics of the
	     software system. In many cases UML is used to represent software
	     architectures.

	\item[Level 4 - requirements] Requirements usually describe intended
	     functionality and attributes of a system from the users point of
	     view.

	\item[Level 5 - product lines] "A software product line is a family of
	     products that share common features to meet the needs of a market
	     area"\footcite[][]{ardis_et_al_2000}
	     Product line documentation covers commonalities and differences in
	     this product line.
\end{description}


\subsection{Relevance of documentation components}
%-----------------------------------------------------
\label{sec:relevance_of_documentation}

In order to define requirements for a documentation system, it is of importance to
define which documentation artifacts are needed to which extend and actuality.

There are several studies about the needs of software maintainers.
The partially very different results and recommendations show how difficult it
is to predict what documents will be needed.

While \citeauthor{tilley_1992} stresses out the general importance of documented
system architecture\footcite[Cf.][]{tilley_1992}, 
\citeauthor{cioch_96} mentions that relevance depends on the
maintainers stage of experience (from new comer to expert; after some years of
working on the system). According to their study new comers need a general
overview of the system and experts need low-level documentation such as API
descriptions.\footcite[Cf.][]{cioch_96}

According to a survey by \citeauthor{forward_relevance_2002} among managers and
maintainers, specification documents are most consulted whereas low level
documents are least consulted\footcite[Cf.][28-30]{forward_relevance_2002}.

\citeauthor{grubb2003software} map information needs to maintainers roles
in the software project. Analysts need to have a global view of the system and
know the requirements. Designers need architectural understanding and
information about low-level components. Finally programmers need a deep
understanding of the source code and a higher level view - comparable to level
two and three above.\footcite[Cf.][103-106]{grubb2003software}

A survey by \citeauthor{de_souza_study_2005} offers a more differentiated view.
Software maintainers were asked to answer how often they use different documents
and how important they think these documents are. Answers of maintainers who
said they work in projects with structured analysis paradigm (70 maintainers)
were separated of those who said they work in projects with an object-oriented
approach (54 maintainers).\footcite[Cf.][69-74]{de_souza_study_2005}
According to the results of the survey, the source code itself is most important
and comments get ranked second. Contrary to the results of \citeauthor{forward_relevance_2002}
specification documents are only ranked on places seven to ten.
One reason for these contrary results might be different project sizes, stages
and environments.


\FloatBarrier %nothing should float out of this subsubsection
\newpage
%------------------------------------------
\section{Documentation system requirements}
%------------------------------------------

\subsection{Dependencies of requirements}

Aenean eu dui a magna tempor dignissim. In sagittis auctor justo, id laoreet
augue ultricies at. Nunc convallis ullamcorper viverra. Sed quis vulputate
erat. Class aptent taciti sociosqu ad litora torquent per conubia nostra, per
inceptos himenaeos. Etiam posuere lacus ut nisl lobortis eu consectetur risus
tempus. Ut mollis, est et posuere porttitor, lacus dui congue libero, et
rhoncus massa ipsum a felis. 

\subsection{Requirements for an exemplary small-scale project}
\label{sec:requirements}


Reference to table \ref{tab:importance_artifacts}.\\


\begin{table}[H]
\begin{center}
  \begin{tabular}{p{2cm}|p{8cm}}
	Priority &	Documentation artifact\\\hline
	1. &	General project information \\
	2. &	Requirements \\
	3. &	Low level documentation\\
	4. &	System architecture\\
  \end{tabular}
	\captionof{table}{caption}
	\label{tab:importance_artifacts}
\end{center}
\end{table}




\newpage
%------------------------------------
\section{Selection of tools}
%------------------------------------


\subsection{Revision control}
%=============================

Fusce pellentesque mattis neque, vitae porta sapien ornare quis. Etiam mollis
magna vitae velit consequat sed blandit nisi pulvinar. Ut nec mi at felis
adipiscing porttitor eget sit amet velit. Duis rutrum\\
nulla ut turpis semper eget congue leo eleifend. Aliquam fermentum, diam ac
facilisis rhoncus, nunc eros aliquet nisi, eget mattis massa velit sit amet
mi. Curabitur vitae eleifend massa. In ut nisi justo, in sodales mi. Aliquam a
nisl ligula, eu ornare nibh. Duis lobortis dui at nibh faucibus bibendum.
Suspendisse potenti. Proin semper urna at neque consequat fermentum.

\begin{itemize}
	\item Timestamp of the change
	\item File/s that have been changed
	\item Editor of the file/s
	\item reference to bug-report if applicable
\end{itemize}

In general version control systems (VCS) can be categorized in two different
types.\\
One of them is classical VCS where there is a central repository that gets
updated by all its committers and the other one is Distributed VCS (DVCS) where
repositories can be distributed on different hosts and clients.

In PEP 374\footcite{pep_374} several Python developers compare VCS and DVCS with
common use case scenarios.\\



%***********
\subsection{Documentation generators}
Aenean eu dui a magna tempor dignissim. In sagittis auctor justo, id laoreet
augue ultricies at. Nunc convallis ullamcorper viverra. Sed quis vulputate
erat. Class aptent taciti sociosqu ad litora torquent per conubia nostra, per
inceptos himenaeos. Etiam posuere lacus ut nisl lobortis eu consectetur risus
tempus. Ut mollis, est et posuere porttitor, lacus dui congue libero, et
rhoncus massa ipsum a felis. 


%Preselection explanation
A preselection based on the above criteria resulted in a few representative
documentation tools with a strong user basis and ongoing development.


\begin{itemize}
	\item Docutils\footnote{Project website:
	   \url{http://docutils.sourceforge.net/}} was picked as a generic
	   documentation tool, that is written for general
	   purpose. %http://docutils.sourceforge.net/
	\item Sphinx\footnote{Project website: \url{http://sphinx.pocoo.org}} is
	   a full featured documentation server that was originally written for
	   the Python
	   language but also supports other domains. %http://sphinx.pocoo.org
	\item Doxygen\footnote{Project website:
	   \url{http://www.stack.nl/~dimitri/doxygen/}} is a documentation
	   system that supports many domains and is specialized on extracting
	   documentation directly from the source
	   code.\\%http://www.stack.nl/~dimitri/doxygen/
\end{itemize}





\newpage
%-----------------------------------------------------------------
\section{Design and implementation of the documentation system}
%----------------------------------------------------------------
% summary of selected tools
% reference to the overall requirements
% theoretical approach
% practical approach


%---------
\subsection{Building the subsystems}
After selecting the right tools, they need to be installed and configured on the
server.\\ In a second step those subsystems need to be interconnected to a
full-featured documentation system.

\subsubsection{Version control system}

Aenean eu dui a magna tempor dignissim. In sagittis auctor justo, id laoreet
augue ultricies at. Nunc convallis ullamcorper viverra. Sed quis vulputate
erat. Class aptent taciti sociosqu ad litora torquent per conubia nostra, per
inceptos himenaeos. Etiam posuere lacus ut nisl lobortis eu consectetur risus
tempus. Ut mollis, est et posuere porttitor, lacus dui congue libero, et
rhoncus massa ipsum a felis. 

\begin{lstlisting}[breaklines=true,frame=single,caption={WebDAV configuration in /etc/apache2/mods-available/dav\_svn.conf},label=lst:davsvn]
<Location /svn/umbrella>
  DAV svn
  SVNPath /var/svn/umbrella
  AuthType Basic
  AuthName "Subversion Repository"
  AuthUserFile /etc/apache2/dav_svn.passwd
  Require valid-user
</Location>
\end{lstlisting}

Reference to listing \ref{lst:davsvn}.



\subsubsection{Documentation generator}
\label{sec:documentation_generator}

Another listing

\lstset{tabsize=2,language=python}
\begin{lstlisting}[breaklines=true,frame=single,caption={Python file with doc-strings},label=lst:doc-string]
'''
Defines all functions that are needed for creating the zip-file.
'''
from django.template.loader import render_to_string

def render_to_file(template, filename, context):
    '''
    Renders a template with the given dictionary as
    context.
    Uses ``django.template.loader.render_to_string``

    :param template: The template_name may be a string to load a single template using
       get_template, or it may be a tuple to use select_template to find one of
       the templates in the list.       
    :param filename: Filename/path to use for saving the rendered template       
    :param context: Dictionary that serves as context

    :rtype: None
    '''
    open(filename, "w").write(render_to_string(template, context))
\end{lstlisting}

\begin{figure}[H]
\begin{center}
	\fbox{
	\includegraphics[width=\textwidth]{img/placeholder1.jpg}
	}
	\captionof{figure}{Representation of a method in the API documentation in HTML format}
	\label{fig:api_render_to_file}
\end{center}
\end{figure}

In order to be able to automatically create the API documentation in more than
one format an additional shell-script was written, which is shown in listing
\ref{lst:builddocs}.

\lstset{tabsize=2}
\begin{lstlisting}[breaklines=true,frame=single,caption={Shellscript for building the software documentation in HTML and PDF},label=lst:builddocs]
#!/bin/bash
sphinx-apidoc -f -o source/api/apps ../umbrella/apps
echo "building html docs..." > build.log
annotate-output make html >> build.log
echo "building pdf doc with latex..." >> build.log
annotate-output make latexpdf >> build.log
\end{lstlisting}

Fusce id magna leo, in porta lectus. Aenean convallis magna sit amet ligula
porttitor vehicula. Morbi pulvinar justo sed sapien pellentesque vitae viverra
dui hendrerit. Vivamus nulla nunc, imperdiet congue euismod ultricies,
sollicitudin eu felis. Pellentesque commodo risus eu quam facilisis et semper
dolor aliquam. Vivamus neque magna, mattis vel aliquet eu, vulputate et leo.
Proin congue dui eget dolor tempus vel congue magna ultrices. Aliquam ut elit
ac lorem molestie adipiscing. Duis at nulla et lorem condimentum faucibus.
Integer commodo turpis id erat volutpat consectetur. Mauris tempor gravida
dui, vitae hendrerit quam laoreet sed. Vivamus eu lorem vel massa pulvinar
pretium. Aenean dignissim porta arcu, ac pretium tortor placerat porttitor. 


%----------
\subsection{Combination of the subsystems}



\lstset{tabsize=2,language=bash}
\begin{lstlisting}[breaklines=true,frame=single,caption={Shell-script that handles the complete build process of the documentation},label=lst:cron-job]
#!/bin/bash

cd /home/umbrella/umbrella/trunk/umbrella
svn update
cd ../docs/source
svn update
cd ..
./build_docs.sh
\end{lstlisting}




\newpage
%-------------------------------------
\section{Conclusion}
%------------------------------------------
Fusce id magna leo, in porta lectus. Aenean convallis magna sit amet ligula
porttitor vehicula. Morbi pulvinar justo sed sapien pellentesque vitae viverra
dui hendrerit. Vivamus nulla nunc, imperdiet congue euismod ultricies,
sollicitudin eu felis. Pellentesque commodo risus eu quam facilisis et semper
dolor aliquam. Vivamus neque magna, mattis vel aliquet eu, vulputate et leo.
Proin congue dui eget dolor tempus vel congue magna ultrices. Aliquam ut elit
ac lorem molestie adipiscing. Duis at nulla et lorem condimentum faucibus.
Integer commodo turpis id erat volutpat consectetur. Mauris tempor gravida
dui, vitae hendrerit quam laoreet sed. Vivamus eu lorem vel massa pulvinar
pretium. Aenean dignissim porta arcu, ac pretium tortor placerat porttitor. 




%-----------------------------------
%END CONTENT
%-----------------------------------


% include footer with appendix, affidavit,...
%----------------------------------------------------------------------------
% APPENDIX
%----------------------------------------------------------------------------
\newpage \begin{appendices} %sets the appendix environment and resets the
							% section counters
				% \renewcommand{\appendixtocname}{List of appendices}
				% \renewcommand{\appendixpagename}{List of appendices}
\appendixtocon %adds an 'Appendices' entry to the toc
\renewcommand*\appendixpagename{\Large Appendices}  %renewal needed to set
													% proper size of title
\appendixpage %prints the title on the page

\subsection*{List of Appendices}
% TODO: create list of appendices with same style as lot or lof


\begin{subappendices}
% \appendixtocoff cant be used here.
% Need to turn toc off for whole environment and manually set the toc entry
% with \addcontentsline{toc}{subsection}{Preface}

\renewcommand{\setthesubsection}{\arabic{subsection}:}%


\subsection{something}
kskfslfrlerg wre gergh  ht rh r h rh   hrthrhrtjukseweffre htzkjtsht erthzrj
kskfslfrlerg wre gergh  ht rh r h rh   hrthrhrtjukseweffre htzkjtsht erthzrj
kskfslfrlerg wre gergh  ht rh r h rh   hrthrhrtjukseweffre htzkjtsht erthzrj
kskfslfrlerg wre gergh  ht rh r h rh   hrthrhrtjukseweffre htzkjtsht erthzrj
kskfslfrlerg wre gergh  ht rh r h rh   hrthrhrtjukseweffre htzkjtshtks
kfslfrlerg\\
wre gergh  ht rh r h rh   hrthrhrtjukseweffre htzkjtsht erthzrj erthzrj

\subsection{something else}
najsjdn fwfwegt eg r th er5h z5er6h  rhg 

\end{subappendices}
\end{appendices}
%----------------------------------------------------------------------------
% BIBLIOGRAPHY
%----------------------------------------------------------------------------
% bibliography file is set in header.tex

\newpage
\addsec{Lists of references} %addsec: komaskript befehl - keine Nummerierung

% Set headings for bibliographies
\defbibheading{literatur}{\subsection*{List of Literature}}
\defbibheading{www}{\subsection*{List of Internet and Intranet Resources}}

% Show bibliographies
\printbibliography[nottype=online,heading=literatur]
% warning: includes ALL types but online.
% Check your bibfile to avoid wrong entries here
\printbibliography[type=online,heading=www]


%----------------------------------------------------------------------------
% Affidavit
%----------------------------------------------------------------------------
\newpage
\pagestyle{empty}

\section*{Affidavit}

\vspace{0.5cm}

"I hereby declare\\

\begin{enumerate}
	\item that I completed my project paper without external help;
	\item that I indicated the use of literal citations from literature as
	      well as the use of ideas of other authors at the respective
	      places in the paper.
	\item that I have not presented my project paper for any other
	      examination.
\end{enumerate}

I am aware of the fact that an untrue declaration will have legal
consequences."
\\ \\ \\ \\
\_\_\_\_\_\_\_\_\_\_\_\_\_\_\_\_\_\_\_\_\_\_\_\_\\
PLACE, DD/MM/YYYY


\end{document}

